A prototype High Pressure gas Time Projection Chamber built at Royal Holloway, University of London, was operated at CERN in the T10 beamline. The vessel was exposed to a 0.8~GeV/c beam of protons in order to make measurements of low momentum protons in argon. The dual technique of placing acrylic moderator in the beamline and placing the vessel off the beam axis was used to decrease the momentum of protons measured and change the flux of protons and other particles reaching the time projection chamber. Measurements of properties of the beam were made using time of flight systems upstream and downstream of the vessel. The momentum range of reaching the vessel was successfully changed from 0.55-0.65~GeV/c with 0~moderator blocks to 0.30-0.45~GeV/c with 4~moderator blocks. The flux of both protons and minimum ionising particles was increased off the beam axis, with the ratio of the two produced varying as a function of off axis angle. Detailed results of the measurements made of the beam are presented.

