\section*{Abstract}
We present studies of proton fluxes in the T10 beamline at CERN.
A prototype high pressure gas time projection chamber was exposed to the beam of protons and other particles, using the 0.8~GeV/c momentum setting in T10, in order to make cross section measurements of low energy protons in argon.
To explore the energy region comparable to hadrons produced by GeV neutrino interactions at oscillation experiments, i.e. near 0.1~GeV of kinetic energy, methods of moderating the T10 beam were employed:
the dual technique of moderating the beam with acrylic blocks and measuring scattered protons off the beam axis was used to decrease the kinetic energy of incident protons, as well as change the p/$\pi$/$\mu$ composition of the incident flux.
Measurements of of the beam properties were made using time of flight systems upstream and downstream of the TPC. 
The kinetic energy of protons reaching the TPC was successfully changed from $\sim$0.3~GeV without moderator blocks to less than 0.1~GeV with four moderator blocks (40~cm path length).
The flux of both protons and minimum ionising particles off the beam axis was increased, with the ratio of the two varying as a function of off-axis angle. 
Simulation informed by the time of flight measurements show that with four moderator blocks placed in the beamline,  ($5.6 \pm 0.1$) protons with energies below 0.1~GeV per spill traversed the active TPC region.
Measurements of the beam composition and energy are presented.
