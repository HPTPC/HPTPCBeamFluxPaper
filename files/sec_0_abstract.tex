A prototype High Pressure gas Time Projection Chamber was operated at CERN in the T10 beamline. The vessel was exposed to a beam of protons of 0.8~GeV/c momentum in order to make measurements of low momentum protons in argon. JOCELYN:"this abstract doesn't give a sense of why we want to do this-- need to add a sentence following sentence 2 about the relevant low momentum range for neutrino oscillation experiments, and wanting to explore producing this low momentum range in the CERN test beams.  This paper presents the results of attempting to moderate the PS test beam down to lower proton momenta." The dual technique of placing acrylic moderator in the beamline and placing the vessel off the beam axis was used to decrease the momentum of protons measured and change the proton/$\pi$/$\mu$ composition of the incident flux. Measurements of properties of the beam were made using time of flight systems upstream and downstream of the vessel. The momentum range of protons reaching the vessel was successfully changed from 0.55-0.65~GeV/c with 0~moderator blocks to 0.30-0.45~GeV/c with 4~moderator blocks. The flux of both protons and minimum ionising particles was increased off the beam axis, with the ratio of the two produced varying as a function of off axis angle. Detailed results of the beam composition measurements are presented.

