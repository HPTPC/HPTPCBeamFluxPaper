\section*{Abstract}

A prototype High Pressure gas Time Projection Chamber was operated at CERN in the T10 beamline. The vessel was exposed to a beam of protons, using the 0.8~GeV/c momentum setting in T10, in order to make measurements of low energy protons in argon.
To explore the energy region comparable to hadrons produced by neutrino interactions at oscillation experiments, i.e. near 100~MeV, methods of moderating the T10 beam were employed:
the dual technique of placing acrylic blocks in the beam path and placing the vessel off the beam axis was used to decrease the kinetic energy of protons measured, as well as change the p/$\pi$/$\mu$ composition of the incident flux.
Measurements of properties of the beam were made using time of flight systems upstream and downstream of the vessel. 
The kinetic energy range of protons reaching the vessel was successfully changed from $\sim$0.3~GeV with 0~moderator blocks to less than 0.1~GeV with 4~moderator blocks.
The flux of both protons and minimum ionising particles was increased off the beam axis, with the ratio of the two produced varying as a function of off-axis angle. 
Simulation informed by the time of flight measurements show that with 4~moderator blocks placed in the beamline,  ($5.56 \pm 0.10$) protons with energies below 100~MeV per spill traversed the active TPC region
Measurements of the beam composition and energy are presented.

