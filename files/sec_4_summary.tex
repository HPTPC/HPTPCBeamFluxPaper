\section{Conclusion}
\label{hptpcPaper:sec:Conclusion}

The prototype High Pressure gas Time Projection Chamber was operated in the T10 beamline at CERN in August and September 2018 in order to make measurement of low momentum protons in argon.
The vessel was placed at a position off the axis of the beam, and a number of moderator blocks were placed directly in the beamline in order to produce a flux of low momentum protons through the TPC, ensure a low occupancy of these low energy protons within the TPC and change the ratio of MIPs to protons off the beam axis.
Measurements of the beam flux were made using two time of flight systems placed ($1.3225 \pm 0.0014$)~m upstream and ($0.9175 \pm 0.0014$)~m downstream of the TPC vessel.
These measurements were used to determine the absolute and relative rates of protons and MIPS as well as their momenta, at different positions off the beam axis, and for varying numbers of moderator blocks.

These measurements demonstrated that the unmoderated T10 beam is broader and has a proton momentum range in line with the 0.8~GeV/c expected.
In line with MC simulation results, the addition of moderator blocks, the average momentum of protons reaching the TPC was reduced from 0.55-0.65~GeV/c with 0~moderator blocks to 0.30-0.45~GeV/c for 4 moderator blocks.
%The ratio of protons to MIPs was not uniformly increased as expected for a more focused beam.
While both the MIP and proton ratio were increased at off-axis positions, the overscattering of protons in the unexpectedly diffuse beam caused their increase in number to be smaller than that of the MIPs for some angles.
Given the number of particles observed by the downstream time of flight with 4~moderator blocks in the beamline, it is expected that [PROTON STEEL RESULT] protons per spill would reach the TPC active region with a momentum range of.
\todo[inline]{Toby to add MC study results and plots}

% classical summary/outlook
