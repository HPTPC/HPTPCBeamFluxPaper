\section{Introduction}

The High Pressure gas Time Projection Chamber (HPTPC) prototype built at Royal Holloway, University of London, underwent a beam test at the T10 beamline in CERN in August and September 2018.

A primary objective of the test was to make measurements of low momentum protons in Argon, and a novel technique was used to achieve this.
The experiment combined use of moderator blocks in the beamline, and making measurements a few degrees off the beam axis.
These techniques increased the ratio of protons to the beam's pion background in the TPC, while also decreasing the proton momentum.

Crucial to the evaluation of this method, was the use of two time of flight (ToF) systems, upstream and downstream of the TPC.
These systems were used to make measurements of flux and proton pion ratio of particles traversing the TPC, as a function of the off-beam axis angle.
A detailed description of the time of flight systems is given below, followed by an analysis of the results they provided over the course of the beam test.