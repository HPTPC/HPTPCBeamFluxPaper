\section{Introduction}

One of the major goals of the global neutrino physics programme is to explore why we live in a matter-dominated universe. Charge-parity symmetry violation (CPV) in the neutrino sector is one of the few remaining possibilities to be explored experimentally. CPV searches compare the prbabilities of $\nu_{\mu}\!\rightarrow\!\nu_e$ and $\overline{\nu}_{\mu}\!\rightarrow\!\overline{\nu}_e$ oscillation, which, in the absence of CPV, should be equal. To convert the measured rate of interactions to a level of CPV, experiments must accurately know the cross section for the interaction with the detector of both neutrinos and antineutrinos. Therefore, systematic uncertainties on neutrino-nucleus interaction cross sections are a key input to such CPV searches.  These interaction cross-sections are dependent on theoretical models because the target nucleon resides in a complicated nucleus, and the nuclear model has dramatic effects on the measured final-state particle kinematic distributions.

The current world’s best accelerator neutrino oscillation experiment, T2K, reports neutrino interaction systematic uncertainties at the XXX\% level [XXX].  The future DUNE and Hyper-K projects assume systematic errors at the 1-2\% level to achieve their physics goals, but will be in the same position as current experiments (>5\% errors) if the nuclear-model uncertainties are not reduced.  The key to reducing these uncertainties is to precisely measure the multiplicity and momentum distribution of final-state particles. However these distributions are modified by final state interactions (FSI) of the recoiling secondary particles as they leave the target nucleus.  The most commonly used neutrino generator Monte Carlos, NEUT and GENIE, simulate FSI with cascade models that are tuned with external hadron-nucleus scattering measurements.  However, as shown in Figure XXX, these proton-nucleus (and pion-nucleus) scattering measurements are extremely sparse and in many cases do not exist in the relevant momentum region and/or on the relevant nuclei.  Therefore semi-empirical parameterisations are used to extrapolate in momentum and atomic mass.  These parametrisation are different between NEUT and GENIE, and yield order-of-magnitude scale differences in the predicted multiplicity and kinematics of final state protons.  The proton final state modeling is a key ingredient for neutrino oscillation measurements because it affects the event selection and neutrino energy reconstruction in charged-current interactions, which is the channel used to measure oscillation parameters and therefore central to the search for CP violation.  For these reasons, FSI contribute dominantly to the total neutrino interaction systematic uncertainty [Abe2014].

Further, FSI models are in tension with data.  Recent neutrino scattering measurements have shown that the most-used models of neutrino-nucleus interactions (employed by NEUT and GENIE) differ from nature in both cross section and kinematics of final state particles by as much as 30\% [XXX]. These errors cannot yet be mitigated with near / far detector combinations because they come from theoretical model deficiencies [Coloma2013] that are not cancelled in the ratio. To achieve 1-2\% systematics for CPV searches, interaction models must be tuned and validated against precise proton-nucleus and pion-nucleus scattering measurements covering a broad final-state particle phase space, on a range of nuclear targets [Cao2014].

Gas TPCs are ideal for such measurements because of the 4$\pi$ angular coverage of all final state particles with high efficiency and low momentum thresholds, which are the keys to distinguish between models.  Figure XXX shows the proton multiplicity and momentum distributions for $\nu_{\mu}$ charged-current (CC) interactions on Argon calculated by the NEUT and GENIE neutrino generators.  These distributions are highly discrepant, particularly in the fraction of events with few ejected protons, and at low proton momentum, below 200 MeV/c. This is below the proton detection threshold in water Cherenkov detectors (1100 MeV/c) and liquid Argon TPCs (200 MeV/c). 

In the HPTPC prototype we have built, the threshold for a well-reconstructed proton in Argon at 5 (10) bar is 53 MeV/c (68 MeV/c), and therefore such a detector could accurately probe the key low-momentum region of parameter space to reduce neutrino interaction cross section uncertainties. This measurement range is highly complementary to what will be learned from the liquid Argon DUNE prototype beam test, above 200 MeV/c threshold. The Co-Is will develop the cross section analysis tools for this measurement such that a consistent approach is applied across the low momenta proposed here and the higher momenta DUNE beam test data (Section 3.X).  As the DUNE prototype beam test Physics Coordinator, Jaroslaw Nowak will synchronise this activity. 	The High Pressure gas Time Projection Chamber (HPTPC) prototype built at Royal Holloway, University of London, underwent a beam test at the T10 beamline in CERN in August and September 2018 \cite{SPSC-P-355}.

A primary objective of the test was to make measurements of low momentum protons in Argon, and a novel technique was used to achieve this.
The experiment combined use of moderator blocks in the beamline, and making measurements a few degrees off the beam axis.
These techniques were designed to increase the ratio of protons to the beam's minimum ionizing particle (MIP) background in the TPC, while also decreasing the proton momentum, and multiplicity in the active region.

Crucial to the evaluation of this method, was the use of two time of flight (ToF) systems, upstream and downstream of the TPC.
These systems were used to make measurements of the flux and proton-MIP ratio of particles traversing the TPC, as a function of the off-beam axis angle.
A detailed description of the time of flight systems is given below, followed by an analysis of the results they provided over the course of the beam test.