\section{Conclusion}
\label{hptpcPaper:sec:Conclusion}

The prototype high pressure gas time projection chamber was operated in the T10 beamline at CERN in August and September 2018 in order to make measurement of low momentum protons in argon.
The vessel was placed at a position off the centre axis of the beam, and a number of acrylic blocks were placed directly in the beamline in order to produce a flux of low momentum protons through the TPC, ensure a low occupancy of these low energy protons within the TPC and change the ratio of MIPs to protons.
Measurements of the beam flux were made using two time of flight systems placed ($1.323 \pm 0.001$)~m upstream and ($0.918 \pm 0.001$)~m downstream of the TPC vessel.
These measurements were used to determine the absolute and relative rates of protons and MIPs as well as their momenta, at different positions off the beam axis, and for varying numbers of moderator blocks.

These measurements demonstrated that adding moderator blocks reduced the average kinetic energy of protons reaching the TPC from 0.3~GeV with 0~moderator blocks to 0.1~GeV for 4 moderator blocks, accessing the kinematic region of interest.
%The ratio of protons to MIPs was not uniformly increased as expected for a more focused beam.
The proton/MIP ratio increased at low off-axis angles, peaking at 1–2 degrees off axis, depending on how many moderator blocks were used, and then fell off at higher angles.
The four moderator block configuration yielded a proton/MIP ratio that was  substantially lower than 0–3 blocks and also flat versus off-axis angle, but achieved the desired proton energy spectrum.
With calibration from the upstream and downstream time of flight systems, for data with 4 moderator blocks in the beamline the simulated number of protons with energy below 100~MeV reaching the active TPC region was ($5.6 \pm  0.1$) per spill with an energy range of 0 to 50~MeV/c.


% classical summary/outlook
