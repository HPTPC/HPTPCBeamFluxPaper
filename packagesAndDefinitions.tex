\usepackage{amsfonts}
\usepackage{amsmath}           %allows the use of align
\usepackage{amssymb}
\usepackage{amstext}
\usepackage[affil-it]{authblk}
\usepackage[UKenglish]{babel}
\usepackage[backref=false,backend=bibtex,style=numeric-comp,sorting=none,maxcitenames=1,block=none,giveninits,url=false]{biblatex}
\addbibresource{hptpc-instrumentation-paper.bib}
\usepackage{bm}                %bold in math mode
\usepackage{booktabs}
\usepackage{color}
\usepackage{enumerate}
\usepackage{fancyhdr}
\usepackage{float}
\usepackage{graphicx}
\usepackage[space]{grffile}    %allows spaces in picture names
\usepackage[top=2.5cm, right=2.5cm, left=2.5cm, bottom=2.5cm]{geometry}
\usepackage[colorlinks=false, pdfborder={0 0 0}, breaklinks]{hyperref}
%\usepackage[latin1]{inputenc}
\usepackage[utf8]{inputenc}
\usepackage[switch]{lineno}    %line numbers, switch: left right line numbers %,columnwise
\usepackage{longtable}
\usepackage{marvosym}
\usepackage{multicol}
\usepackage{setspace}
\usepackage[separate-uncertainty,multi-part-units = single, allow-number-unit-breaks]{siunitx}
\usepackage{subfig}
\usepackage{tikz-timing}
\usepackage{tikz}
\usepackage{adjustbox}

\usepackage[colorinlistoftodos,prependcaption,textsize=tiny]{todonotes} % Add TODO notes to the text
%\usepackage{ucs}

\usepackage{setspace}
\doublespacing
\usepackage{url}
\def\UrlBreaks{\do\/\do-}
\modulolinenumbers[5]

\usepackage{csquotes}

%Remove urls and dois
\AtEveryBibitem{%
  \clearfield{issn} % Remove issn
  \clearfield{doi}  % Remove doi

  \ifentrytype{online}{}{% Remove url except for @online
    \clearfield{url}
  }
}

%footnotes indicated with letters
\renewcommand*{\thefootnote}{\alph{footnote}}

%adding the abstract to the title environment
\makeatletter
\newbox\abstract@box
\renewenvironment{abstract}
  {\global\setbox\abstract@box=\vbox\bgroup
     \hsize=\textwidth\linewidth=\textwidth
    \small
    \begin{center}%
    {\bfseries \abstractname\vspace{-.5em}\vspace{\z@}}%
    \end{center}%
    \quotation}
  {\endquotation\egroup}
\expandafter\def\expandafter\@maketitle\expandafter{\@maketitle
  \ifvoid\abstract@box\else\unvbox\abstract@box\if@twocolumn\vskip1.5em\fi\fi}
\makeatother

%fix a line numbering problem
\newcommand*\patchAmsMathEnvironmentForLineno[1]{%
  \expandafter\let\csname old#1\expandafter\endcsname\csname #1\endcsname
  \expandafter\let\csname oldend#1\expandafter\endcsname\csname end#1\endcsname
  \renewenvironment{#1}%
     {\linenomath\csname old#1\endcsname}%
     {\csname oldend#1\endcsname\endlinenomath}}% 
\newcommand*\patchBothAmsMathEnvironmentsForLineno[1]{%
  \patchAmsMathEnvironmentForLineno{#1}%
  \patchAmsMathEnvironmentForLineno{#1*}}%
\AtBeginDocument{%
\patchBothAmsMathEnvironmentsForLineno{equation}%
\patchBothAmsMathEnvironmentsForLineno{align}%
\patchBothAmsMathEnvironmentsForLineno{flalign}%
\patchBothAmsMathEnvironmentsForLineno{alignat}%
\patchBothAmsMathEnvironmentsForLineno{gather}%
\patchBothAmsMathEnvironmentsForLineno{multline}%
}


% other definitions - gas mixtures
\newcommand{\arco}{$\textrm{Ar}\textnormal{-}\textrm{CO}_2$}
\newcommand{\arcois}[1]{$\textrm{Ar}\textnormal{-}\textrm{CO}_2$ (#1)}
\newcommand{\arcf}{$\textrm{Ar}\textnormal{-}\textrm{CF}_4$}
\newcommand{\arcfis}[1]{$\textrm{Ar}\textnormal{-}\textrm{CF}_4$ (#1)}
\newcommand{\arn}{$\textrm{Ar}\textnormal{-}\textrm{N}_2$}
\newcommand{\arnis}[1]{$\textrm{Ar}\textnormal{-}\textrm{N}_2$ (#1)}
\newcommand{\arcon}{$\textrm{Ar}\textnormal{-}\textrm{CO}_2\textnormal{-}\textrm{N}_2$}
\newcommand{\arconis}[1]{$\textrm{Ar}\textnormal{-}\textrm{CO}_2\textnormal{-}\textrm{N}_2$ (#1)}
\newcommand{\arch}{$\textrm{Ar}\textnormal{-}\textrm{CH}_4$}
\newcommand{\archis}[1]{$\textrm{Ar}\textnormal{-}\textrm{CH}_4$ (#1)}


% other definitions - Figures, Tables, Sections
\newcommand{\figref}[1]{Figure~\ref{#1}}    % Standard
\newcommand{\figrefbra}[1]{Fig.~\ref{#1}}   % In brackets 
\newcommand{\Figref}[1]{Figure~\ref{#1}}    % Beginning of a section
%
\newcommand{\tabref}[1]{Tab.~\ref{#1}}      % Standard
\newcommand{\tabrefbra}[1]{Fig.~\ref{#1}}   % In brackets 
\newcommand{\Tabref}[1]{Table~\ref{#1}}     % Beginning of a section
%
\newcommand{\secref}[1]{Section~\ref{#1}}   % Standard
\newcommand{\secrefbra}[1]{Sec.~\ref{#1}}   % In brackets
\newcommand{\Secref}[1]{Section~\ref{#1}}   % Beginning of a section
%